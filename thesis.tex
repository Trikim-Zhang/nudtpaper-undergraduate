%!TEX program = xelatex%
%!BIB program = biber%
% 出现任何问题请联系(zhangxinhang19@foxmial.com)
\documentclass[biber,ttf]{nudtpaper}
% biber-参考文献、ttf-字体、anon-盲评、twoside-book
\usepackage{mynudt}

\title{你的题目\\
可以换行}%题目
\author{张三}      %学员姓名
\serialno{2019xxxx}       %学号
\firstmajor{首次任职专业}       %首次任职专业
\major{学历教育专业}       %学历教育专业
\college{xx学院}       %所属学院
\grade{2019级}       %年级
\supervisor{xxx}       %指导教员
\teachtitle{x教授}       %职称
\department{xxxx}       %所属单位

\begin{document}
% 制作封面,生成目录,插入摘要,插入符号列表 \\
% 默认符号列表使用denotation.tex,如果要使用nomencl \\
% 需要注释掉denotation,并取消下面两个命令的注释。 \\
% \cleardoublepage% \\
% \printnomenclature% \\
\maketitle
\frontmatter
\tableofcontents
\listoffigures
\listoftables

%如果不是送审论文,则将true改为false即可
\newif\ifreview\reviewfalse

\midmatter
\input{data/abstract}
\input{data/denotation}

%书写正文,可以根据需要增添章节; 正文还包括致谢,参考文献与成果
\mainmatter
\input{data/chap01}
\input{data/chap02}

\input{data/ack}

\cleardoublepage
\phantomsection
\addcontentsline{toc}{chapter}{参考文献}
\renewcommand{\normalsize}{\wuhao[1.25]} %
\renewcommand{\baselinestretch}{1.35}
\ifisbiber
{\hyphenpenalty=500 %
\tolerance=9900 %
\printbibliography[heading=bibliography, title=参考文献]
}
\else
\bibliographystyle{ref/bstutf8}
\bibliography{ref/refs}
\fi

\appendix
\backmatter
\input{data/appendix01}

\end{document}
